\documentclass[a4paper,11pt]{article}
\usepackage[T1]{fontenc}
\usepackage[utf8]{inputenc}
\usepackage{lmodern}
\usepackage[spanish]{babel}

\title{Monitor de Sistema}
\author{Aguilar Enriquez, Paul Sebastian a.k.a Penserbjorne}

\begin{document}

\maketitle
\tableofcontents

\begin{abstract}
\textbf{Planteamiento del profesor:} Programar un monitor de recursos del sistema operativo (de los recursos que juzguen útiles, interesantes, o que decidan por la razón que sea), pero con la particularidad o requisito de que
debe trabajar de forma concurrente, sincronizando multihilos, multiprocesos o ambos. Empleen los mecanismos de sincronización que consideren adecuados.

\textbf{Propuesta del alumno:} Diseñar e implementar un monitor de (algún/algunos) recurso/recursos del sistema, en este caso se pretende realizar para la plataforma GNU/Linux ...
\end{abstract}

\section{Monitor de Sistema}

\subsection{Definicion}

\subsection{Caracteristicas}

\section{Ejemplos de Monitores de Sistema}

\subsection{GNU/Linux}

\subsection{Mac OS}

\subsection{Windows}

\section{Diseño}

\subsection{Propuesta de Monitor de Sistema}

\subsection{Diseño}

\section{Implementación}

\end{document}
