\documentclass[a4paper,11pt]{article}
\usepackage[T1]{fontenc}
\usepackage[utf8]{inputenc}
\usepackage{lmodern}
\usepackage[spanish]{babel}
%\usepackage{biblatex}
%\usepackage{url}
\usepackage{hyperref}

\title{Monitor de Sistema}
\author{Aguilar Enriquez, Paul Sebastian a.k.a Penserbjorne}

\begin{document}

\maketitle
\tableofcontents

\begin{abstract}
\textbf{Planteamiento del profesor:} Programar un monitor de recursos del sistema operativo (de los recursos que juzguen útiles, interesantes, o que decidan por la razón que sea), pero con la particularidad o requisito de que
debe trabajar de forma concurrente, sincronizando multihilos, multiprocesos o ambos. Empleen los mecanismos de sincronización que consideren adecuados.

\textbf{Propuesta del alumno:} Diseñar e implementar un monitor de (algún/algunos) recurso/recursos del sistema, en este caso se pretende realizar para la plataforma GNU/Linux ...
\end{abstract}

\section{Monitor de Sistema}

Investigando un poco, sobre todo por Internet, es muy complicado encontrar una definicion exacta de lo que es un ''Monitor de Sistema''. Para el usuario intermedio-avanzado de un equipo de computo puede que el termino sea inherente, sin embargo a continuacion dejaremos algunas definiciones que podrian describir un poco lo que es un ''Monitor de Sistema''.

\begin{enumerate}
  \item El monitor del sistema muestra qué programas están en ejecución y cuánto procesador, tiempo, memoria y espacio en disco están usando.\cite{ref:web1}
  
  \item ... es una aplicación integrada en los sistemas operativos ... , gracias a la cual podremos obtener información de los programas y procesos que se ejecutan en el equipo, además de proporcionar los indicadores de rendimientos más utilizados en el equipo.\cite{ref:web2}
  
  \item These tools are primarily divided into two main categories: real time and log-based. Real time monitoring tools are concerned with measuring the current system state and provide up to date information about the system performance. Log-based monitoring tools record system performance information for post-processing and analysis and to find trends in the system performance.\cite{ref:web3}
  
  \item Los procesos que se muestran ... pueden ser apps del usuario, apps del sistema utilizadas o procesos invisibles en segundo plano.\cite{ref:web4}

\end{enumerate}

En general podriamos resumir que un ''Monitor de Sistema'' es un software que monitorea mayormente en tiempo real los procesos e hilos que estan actualmente en ejecucion mostrando usualmente informacion sobre ellos como seria tiempo, memoria y espacio en disco que estan usando, asi como los indicadores de rendimiento mas comunes de un equipo que podrian ser uso de CPU/hardware, memoria, energia, disco y red.

\section{Ejemplos de Monitores de Sistema}

\subsection{GNU/Linux}

\subsection{Mac OS}

\subsection{Windows}

\section{Diseño}

\subsection{Propuesta de Monitor de Sistema}

\subsection{Diseño}

\section{Implementación}

\begin{thebibliography}{}

\bibitem{ref:web1}
  The GNOME Project.
  (-).
  Monitor del sistema.
  23/09/2016, 
  de The GNOME Project 
  Sitio web:   \url{https://help.gnome.org/users/gnome-system-monitor/index.html.es}

\bibitem{ref:web2}
  Xataka Winows. 
  (14/07/2014). 
  El administrador de tareas de Windows: qué es y cómo funciona. 
  23/09/2016, de Xataka Winows 
  Sitio web: \url{http://www.xatakawindows.com/bienvenidoawindows8/el-administrador-de-tareas-de-windows-que-es-y-como-funciona}

\bibitem{ref:web3}
  Arik Brooks. 
  (-). 
  Operating System and Process Monitoring Tools. 
  23/09/2016, de Washington University in St. Louis - School of Engineering \& Applied Science 
  Sitio web: \url{http://www.cs.wustl.edu/~jain/cse567-06/ftp/os_monitors/}

\bibitem{ref:web4}
  Apple. 
  (-). 
  Cómo usar Monitor de Actividad en la Mac. 
  23/09/2016, de Apple 
  Sitio web: \url{https://support.apple.com/es-mx/HT201464}

\end{thebibliography}

Biblioteca ncurses, C
\end{document}
